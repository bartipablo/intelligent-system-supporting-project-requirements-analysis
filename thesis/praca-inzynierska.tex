%%%%%% -*- Coding: utf-8-unix; Mode: latex

\documentclass[polish]{aghengthesis}
%\documentclass[english]{aghengthesis} %dla pracy w języku angielskim. Uwaga, w przypadku strony tytułowej zmiana języka dotyczy tylko kolejności wersji językowych tytułu pracy. 
% Szablon przystosowany jest do druku dwustronnego. 

\usepackage[utf8]{inputenc}
\usepackage{url}
\usepackage{subfigure}
\usepackage{tabularx}
\usepackage{ragged2e}
\usepackage{booktabs}
\usepackage{multirow}
\usepackage{grffile}
\usepackage{indentfirst}
\usepackage{caption}
\usepackage{listings}
\usepackage[ruled,linesnumbered,lined]{algorithm2e}
\usepackage[bookmarks=false]{hyperref}

\hypersetup{colorlinks,
  linkcolor=blue,
  citecolor=blue,
  urlcolor=blue}

\usepackage[svgnames]{xcolor}
\usepackage{inconsolata}

\usepackage{csquotes}
\DeclareQuoteStyle[quotes]{polish}
  {\quotedblbase}
  {\textquotedblright}
  [0.05em]
  {\quotesinglbase}
  {\fixligatures\textquoteright}
\DeclareQuoteAlias[quotes]{polish}{polish}

\usepackage[nottoc]{tocbibind}

\usepackage[
style=numeric,
sorting=nyt,
isbn=false,
doi=true,
url=true,
backref=false,
backrefstyle=none,
maxnames=10,
giveninits=true,
abbreviate=true,
defernumbers=false,
backend=biber]{biblatex}
\addbibresource{bibliografia.bib}

\lstset{
    %language=Python, %% PHP, C, Java, etc.
    basicstyle=\ttfamily\footnotesize,
    backgroundcolor=\color{gray!5},
    commentstyle=\it\color{Green},
    keywordstyle=\color{Red},
    stringstyle=\color{Blue},
    numberstyle=\tiny\color{Black},    
    % morekeywords={TestKeyword},
    % mathescape=true,
    escapeinside=`',
    frame=single, %shadowbox, 
    tabsize=2,
    rulecolor=\color{black!30},
    title=\lstname,
    breaklines=true,
    breakatwhitespace=true,
    framextopmargin=2pt,
    framexbottommargin=2pt,
    extendedchars=false,
    captionpos=b,
    abovecaptionskip=5pt,
    keepspaces=true,            
    numbers=left,                    
    numbersep=5pt,                  
    showspaces=false,                
    showstringspaces=false,
    showtabs=false,
    tabsize=2
  }

\SetAlgorithmName{\LangAlgorithm}{\LangAlgorithmRef}{\LangListOfAlgorithms}
\newcommand{\listofalgorithmes}{\tocfile{\listalgorithmcfname}{loa}}

\renewcommand{\lstlistingname}{\LangListing}
\renewcommand\lstlistlistingname{\LangListOfListings}

\renewcommand{\lstlistoflistings}{\begingroup
\tocfile{\lstlistlistingname}{lol}
\endgroup}

% Definicje nowych rodzajów kolumn w tabeli
\newcolumntype{Y}{>{\small\centering\arraybackslash}X}
%\newcolumntype{b}{>{\hsize=1.6\hsize}Y}
%\newcolumntype{m}{>{\hsize=.6\hsize}Y}
%\newcolumntype{s}{>{\hsize=.4\hsize}Y}

\captionsetup[figure]{skip=5pt,position=bottom}
\captionsetup[table]{skip=5pt,position=top}

%%%%%%%%%%%%%%%%%%%%%%%%%%%%%%%%%%%%%%%%%%%%%%%%%%%%%%%%%%%%%%%%%%%%%%%%%%%%%%%
\author{Jan Kowalski, Jan Malinowski, Wojciech Kowalski}

\titlePL{Społecznościowy system wspomagający zarządzanie odtwarzaniem muzyki w obiektach usługowych i użyteczności publicznej}
\titleEN{A community system supporting the management of music playback in service and public facilities}

\fieldofstudy{Informatyka}

%\typeofstudies{Stacjonarne}

\supervisor{dr hab.\ inż.\ Krzysztof Iksiński, prof.\ AGH}

\date{\the\year}

%%%%%%%%%%%%%%%%%%%%%%%%%%%%%%%%%%%%%%%%%%%%%%%%%%%%%%%%%%%%%%%%%%%%%%%%%%%%%%%
\begin{document}

\maketitle

\tableofcontents

%%%%%%%%%%%%%%%%%%%%%%%%%%%%%%%%%%%%%%%%%%%%%%%%%%%%%%%%%%%%%%%%%%%%%%%%%%%%%%%
\chapter{\ChapterTitleProjectVision}
\label{sec:cel-wizja}

% poniższą zawartość rodziału należy usunąć z finalnej wersji pracy.
\emph{Charakterystyka problemu, motywacja projektu (w tym przegląd istniejących rozwiązań prowadząca do uzasadnienia celu prac), wizja produktu i analiza zagrożeń.}

%%%%%%%%%%%%%%%%%%%%%%%%%%%%%%%%%%%%%%%%%%%%%%%%%%%%%%%%%%%%%%%%%%%%%%%%%%%%%%%
\section{Cytowania literatury}
\label{sec:cytowania}

Przykład cytowania literatury~\cite{wilson2009prediction-interday}. Kolejny przykład
cytowania kilku pozycji bibliograficznych~\cite{allen1999using-genetic, zitzler1999evolutionary-algorithms, pictet1995genetic-algorithms, wilhelmstotter2021jenetics, chmaj2015DistributedProcessingApplications}.

%%%%%%%%%%%%%%%%%%%%%%%%%%%%%%%%%%%%%%%%%%%%%%%%%%%%%%%%%%%%%%%%%%%%%%%%%%%%%%%
\section{Listy}
\label{sec:listy}

Lista z elementami:
\begin{itemize}
\item pierwszym,
\item drugim,
\item trzecim.
\end{itemize}

Lista numerowana z dłuższymi opisami:
\begin{enumerate}
\item Pierwszy element listy.
\item Drugi element listy.
\item Trzeci element listy.
\end{enumerate}

%%%%%%%%%%%%%%%%%%%%%%%%%%%%%%%%%%%%%%%%%%%%%%%%%%%%%%%%%%%%%%%%%%%%%%%%%%%%%%%
\chapter{\ChapterTitleScope}
\label{sec:zakres-funkcjonalnosci}

% poniższą zawartość rodziału należy usunąć z finalnej wersji pracy.
\emph{Kontekst użytkowania produktu (aktorzy, współpracujące systemy) oraz specyfikacja wymagań funkcjonalnych i niefunkcjonalnych.}

%%%%%%%%%%%%%%%%%%%%%%%%%%%%%%%%%%%%%%%%%%%%%%%%%%%%%%%%%%%%%%%%%%%%%%%%%%%%%%%
\section{Rysunki, tabele}
\label{sec:rysunki-tabele}

W tekście powinny się znaleźć odnośniki do wszystkich rysunków i tabel
występujących w pracy.

%%%%%%%%%%%%%%%%%%%%%%%%%%%%%%%%%%%%%%%%%%%%%%%%%%%%%%%%%%%%%%%%%%%%%%%%%%%%%%%
\subsection{Rysunki}
\label{sec:rysunki}

Przykładowy odnośnik do rysunku~\ref{fig:ex1}.

\begin{figure}[!htbp]
  \centering
\includegraphics[width=.7\textwidth]{example.pdf}
\caption[Przykładowy rysunek]{Przykładowy rysunek (źródło:
  \cite{author2021title})}
\label{fig:ex1}
\end{figure}
 
W przypadku rysunków można odwoływać się zarówno do poszczególnych części
składowych --- rysunek~\ref{fig:sub1} i rysunek~\ref{fig:sub2}) --- jak i do
całego rysunku~\ref{fig:ex2}.

\begin{figure}[!htbp]
\begin{center}
\subfigure[Tytuł 1]{%
\label{fig:sub1}
\includegraphics[width=0.48\textwidth]{example.pdf}}%
\subfigure[Tytuł 2]{%
\label{fig:sub2}
\includegraphics[width=0.48\textwidth]{example.pdf}}
\end{center}
\caption[Kolejne przykładowe rysunki]{Kolejne przykładowe rysunki (źródło:
  \cite{author2021title})}
\label{fig:ex2}
\end{figure}

%%%%%%%%%%%%%%%%%%%%%%%%%%%%%%%%%%%%%%%%%%%%%%%%%%%%%%%%%%%%%%%%%%%%%%%%%%%%%%% 
\subsection{Tabele}
\label{sec:tabele}

Przykładowa tabela~\ref{tab:ex1}.

\begin{table}[!htbp]
\centering
\caption[Przykładowa tabela]{Przykładowa tabela}
\begin{tabularx}{\columnwidth}{@{}YYYYYYY@{}} \toprule
  & \multicolumn{2}{c}{\small\textbf{Best}} & \multicolumn{2}{c}{\small\textbf{Average}} & \multicolumn{2}{c}{\small\textbf{Worst}} \\ \cmidrule(lr){2-3} \cmidrule(lr){4-5} \cmidrule(lr){6-7}
  \textbf{No.} & \textbf{AB} & \textbf{CD} & \textbf{FE} & \textbf{GH} & \textbf{IJ} & \textbf{KL} \\ \midrule
  \textit{1.} & 10 & 89 & 58 & 244 & 6 & 70 \\  
  \textit{2.} & 15 & 87 & 57 & 147 & 4 & 82 \\
  \textit{3.} & 23 & 45 & 55 & 151 & 2 & 38 \\
  \textit{4.} & 34 & 90 & 55 & 246 & 1 & 82 \\
  \textit{5.} & 56 & 75 & 54 & 255 & 0 & 73 \\ \bottomrule
\end{tabularx}
\label{tab:ex1}
\end{table}


%%%%%%%%%%%%%%%%%%%%%%%%%%%%%%%%%%%%%%%%%%%%%%%%%%%%%%%%%%%%%%%%%%%%%%%%%%%%%%%
\chapter{\ChapterTitleRealizationAspects}
\label{sec:wybrane-aspekty-realizacji}

% poniższą zawartość rodziału należy usunąć z finalnej wersji pracy.
\emph{Przyjęte założenia, struktura i zasada działania systemu, wykorzystane rozwiązania technologiczne wraz z uzasadnieniem ich wyboru, istotne mechanizmy i zastosowane algorytmy.}

%%%%%%%%%%%%%%%%%%%%%%%%%%%%%%%%%%%%%%%%%%%%%%%%%%%%%%%%%%%%%%%%%%%%%%%%%%%%%%%
\section{Wzory matematyczne}
\label{sec:wzory}

% poniższy tekst należy usunąć z finalnej wersji pracy.

Przykład wzoru z odnośnikiem do literatury~\cite{author2021title}:

\begin{equation}
\Omega = \sum_{i=1}^n \gamma_i
\label{eq:sum}
\end{equation}

Przykładowy odnośnik do wzoru~\eqref{eq:sum}.

Przykładowy wzór w tekście $\lambda = \sum_{i=1}^n \delta_i$, bez numeracji.

%%%%%%%%%%%%%%%%%%%%%%%%%%%%%%%%%%%%%%%%%%%%%%%%%%%%%%%%%%%%%%%%%%%%%%%%%%%%%%%
\section{Algorytmy}
\label{sec:algorytmy}

Algorytm~\ref{alg:pseudo-code} przedstawia przykładowy algorytm zaprezentowany w~\cite{fiorio2017algorithm2e}. 

\begin{algorithm}[!htbp]
\SetKwData{Left}{left}\SetKwData{This}{this}\SetKwData{Up}{up}
\SetKwFunction{Union}{Union}\SetKwFunction{FindCompress}{FindCompress}
\SetKwInOut{Input}{input}\SetKwInOut{Output}{output}
\Input{A bitmap $Im$ of size $w\times l$}
\Output{A partition of the bitmap}
\BlankLine
\emph{special treatment of the first line}\;
\For{$i\leftarrow 2$ \KwTo $l$}{
\emph{special treatment of the first element of line $i$}\;
\For{$j\leftarrow 2$ \KwTo $w$}{\label{forins}
\Left$\leftarrow$ \FindCompress{$Im[i,j-1]$}\;
\Up$\leftarrow$ \FindCompress{$Im[i-1,]$}\;
\This$\leftarrow$ \FindCompress{$Im[i,j]$}\;
\If(\tcp*[h]{O(\Left,\This)==1}){\Left compatible with \This}{\label{lt}
\lIf{\Left $<$ \This}{\Union{\Left,\This}}
\lElse{\Union{\This,\Left}}
}
\If(\tcp*[f]{O(\Up,\This)==1}){\Up compatible with \This}{\label{ut}
\lIf{\Up $<$ \This}{\Union{\Up,\This}}
\tcp{\This is put under \Up to keep tree as flat as possible}\label{cmt}
\lElse{\Union{\This,\Up}}\tcp*[h]{\This linked to \Up}\label{lelse}
}
}
\lForEach{element $e$ of the line $i$}{\FindCompress{p}}
}
  \caption[Przykładowy algorytm]{Przykładowy algorytm (źródło: \cite{fiorio2017algorithm2e}).}
  \label{alg:pseudo-code}
\end{algorithm}

%%%%%%%%%%%%%%%%%%%%%%%%%%%%%%%%%%%%%%%%%%%%%%%%%%%%%%%%%%%%%%%%%%%%%%%%%%%%%%% 
\section{Fragmenty kodu źródłowego}
\label{sec:listingi}
Listing~\ref{lst:maximum} przedstawia przykładowy fragment kodu źródłowego.

\begin{lstlisting}[language=Python,float=!htbp,caption={[Przykładowy fragment kodu]Przykładowy fragment kodu (źródło:
  \cite{author2021title})},label=lst:maximum]
# The maximum of two numbers

def maximum(x, y):

    if x >= y:
        return x
    else:
        return y

x = 2
y = 6
print(maximum(x, y),"is the largest of the numbers ", x, " and ", y)

\end{lstlisting}

%%%%%%%%%%%%%%%%%%%%%%%%%%%%%%%%%%%%%%%%%%%%%%%%%%%%%%%%%%%%%%%%%%%%%%%%%%%%%%%
\chapter{\ChapterTitleWorkOrganization}
\label{sec:organizacja-pracy}

% poniższą zawartość rodziału należy usunąć z finalnej wersji pracy.
\emph{Struktura zespołu (role poszczególnych osób), krótki opis i uzasadnienie przyjętej metodyki i/lub kolejności prac, planowane i zrealizowane etapy prac ze wskazaniem udziału poszczególnych członków zespołu, wykorzystane praktyki i narzędzia w zarządzaniu projektem.}

%%%%%%%%%%%%%%%%%%%%%%%%%%%%%%%%%%%%%%%%%%%%%%%%%%%%%%%%%%%%%%%%%%%%%%%%%%%%%%%
\chapter{\ChapterTitleResults}
\label{sec:wyniki-projektu}

% poniższą zawartość rodziału należy usunąć z finalnej wersji pracy.
\emph{Wskazanie wyników projektu (co konkretnie udało się uzyskać: oprogramowanie, dokumentacja, raporty z testów/wdrożenia, itd.), prezentacja wyników i ocena ich użyteczności (jak zostało to zweryfikowane --- np.\ wnioski klienta/użytkownika, zrealizowane testy wydajnościowe, itd.), istniejące ograniczenia i propozycje dalszych prac.}

%%%%%%%%%%%%%%%%%%%%%%%%%%%%%%%%%%%%%%%%%%%%%%%%%%%%%%%%%%%%%%%%%%%%%%%%%%%%%%%
\printbibliography

%%%%%%%%%%%%%%%%%%%%%%%%%%%%%%%%%%%%%%%%%%%%%%%%%%%%%%%%%%%%%%%%%%%%%%%%%%%%%%%
\listoffigures
\listoftables
\listofalgorithmes
\lstlistoflistings

\end{document}
